\errorcontextlines=9
\english
\iflanguage{english}{%
	\retrofrontespizio{This work is subject to the Creative Commons Licence}
	% \DottoratoIn{PhD Course in\space}
	\StrutturaDidattica{Department of }
	\struttura{Control and Computer Engineering}
	\CorsoDiLaureaIn{Master of Science in\space}
	%\NomeMonografia{Bachelor Degree Final Work}
	\TesiDiLaurea{Master Thesis}
	%\NomeDissertazione{PhD Dissertation}
	%\InName{in}
	\CandidateName{Candidate}% or Candidate
	\AdvisorName{Supervisors}% or Supervisor
	\TutorName{Tutor}
	\NomeTutoreAziendale{Internship Tutor}
	%\CycleName{cycle}
	%\NomePrimoTomo{First volume}
	%\NomeSecondoTomo{Second Volume}
	%\NomeTerzoTomo{Third Volume}
	%\NomeQuartoTomo{Fourth Volume}
	%\logosede{logodue}% or comma separated list of logos
	\TitoloListaCandidati{Candidate,Candidate,Candidates,Candidates}
}{}
%%%%%%% Questi comandi è meglio metterli dentro l'ambiente
%%%%%%% frontespizio o frontespizio*, oppure in un file di
%%%%%%% configurazione personale. Si veda la documentazione
%%%%%%% inglese o italiana.
%%%%%%% Comunque i presenti comandi servono per comporre la
%%%%%%% tesi con i moduli di estensione standard del pacchetto
%%%%%%% TOPtesi.

\begin{ThesisTitlePage}
	\ateneo{Politecnico di Torino}
	\titolo{Deep Reinforcement Learning algorithms for autonomous systems}
	\sottotitolo{Design and implementation of a control system for autonomous driving task of a small robot, exploiting state-of-the-art Model-Free Deep Reinforcement Learning algorithms}
	%%%%%%% Corso degli studi
	\corsodilaurea{Computer Engineering (Software Career)}
	%%%%%%% L'eventuale numero di matricola va fra parentesi quadre
	\candidato{Piero \textsc{Macaluso}}[s252894]
	%\secondocandidato{Evangelista \textsc{Torricelli}}[123457]

	%%%%%%% Relatori o supervisori
	%
	\relatore{prof.~Pietro \textsc{Michiardi}}
	\secondorelatore{prof.~Elena \textsc{Baralis}}

	%%%%%%% Seduta dell'esame
	%		\sedutadilaurea{\textsc{October} 2019}
	%%%%%%%% oppure:
	\sedutadilaurea{\textsc{Academic~year} \todomacaluso{2019-2020}}% 
	%%%%%%% Logo della sede
	\logosede{logopolito}% 
\end{ThesisTitlePage}


%%%%%%% Per cambiare l'offset per la rilegatura;
%%%%%%% meno offset c'e', meglio e'
%\setbindingcorrection{3mm}

\sommario

\todomacaluso{Abstract is the last thing to do}



% \paginavuota % funziona anche senza specificare l'opzione classica

%\printglossaries

\ringraziamenti

\todomacaluso{Acknowledgements must be prepared!}


%		\tablespagetrue\figurespagetrue % normalmente questa riga non serve ed e' commentata
\indici

%%%%%%%% Altro esperimento con l'opzione classica
%%%%%%%% Non usare mai anche se qui lo si è fatto!
%%%%%%%% Oltretutto funziona solo se si è specificata la lingua greca fra le opzioni.
%%%%%%%% Commentare fra \ifclassica fino a \fi compresi. 
\ifclassica
	\begin{citazioni}
		\textit{testo testo testo\\testo testo testo}

		[\textsc{G.\ Leopardi}, Operette Morali]\vspace{1em}

		\textgreek{>all'a p'anta <o k'eraunos d'' >oiak'izei}

		[\textsc{Eraclito}, fr.\ D-K 134]
	\end{citazioni}

\fi
%%%%%%%% fine esperimento

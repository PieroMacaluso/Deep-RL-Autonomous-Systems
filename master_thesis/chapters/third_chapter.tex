\chapter{Tools and Frameworks}

\todomacaluso{
	\begin{itemize}
		\item Introduction to the chapter
	\end{itemize}	
}

\section{OpenAI Gym}


\todomacaluso{
	\begin{itemize}
		\item General description of the framework
		\item Discussion about the motivation and the importance of this framework, such as the necessity of a Reinforcement Learning Framework to test different algorithms with different environments providing the same interface.
		\item One contribution of the thesis: the creation and design of an OpenAI Gym environment for Anki Cozmo Environment with connection to chapter 4.
	\end{itemize}	
}

\subsection{The importance of a Framework}



\subsection{Main features}



\subsection{How to create an environment}


\section{Anki Cozmo}

\todomacaluso{
	\begin{itemize}
		\item General description of Cozmo and Anki
		\item General information about the mechanics and features of Cozmo (LINK)
		\item Discussion about the selection of Cozmo instead of other solutions
		\begin{itemize}
			\item Amazon Deep Racer: not available at the start of the thesis. It provides a simulator to train the agent. Using AWS for computation which can be a benefit, but also a drawback because it is a lock-in solution.
			\item Building a Car: one of the best path to follow because of the personalization available. Main drawbacks are the length of the car construction process but also the time to spend in the creation of interfaces between the car and Python.
			\item In the end, Cozmo is the best trade-off between functionalities and fast-developing. It provides plain and straightforward control of the car and a rich Python SDK to use with OpenAI Gym.
		\end{itemize}
		\item Discussion about the on-board or off-board computation
	\end{itemize}	
}



\subsection{Features of Cozmo}



\subsection{Cozmo SDK}




\section{PyTorch}


\todomacaluso{
	\begin{itemize}
		\item General description of Pytorch framework
	\end{itemize}	
}



\subsection{Tensor and Gradients}



\subsection{Building a Convolutional Neural Network}



\subsection{Loss function and Optimizers}



\subsection{TensorboardX}


